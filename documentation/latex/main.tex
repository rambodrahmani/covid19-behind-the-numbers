%-------------------------------------------------------------------------------
% File: main.tex
%	COVID-19 Behind the Numbers project documentation.
%
%	Compile using:
%	    $ pdflatex main.tex
%	    $ biber main
%
% Author: Rambod Rahmani <rambodrahmani@autistici.org>
%	  Created on 07/01/2021
%-------------------------------------------------------------------------------
\documentclass[11pt,a4paper]{article}

\usepackage[a4paper, portrait, margin=1.1in]{geometry}
\usepackage[dvipsnames]{xcolor}
\usepackage[linktoc=none]{hyperref}
\hypersetup{
	colorlinks=true,
	linkcolor=blue,
	filecolor=magenta,      
	urlcolor=blue,
}
\usepackage{listings}
\usepackage{float}
\usepackage{graphicx}
\usepackage[justification=centering]{caption}
\usepackage{wrapfig}
\usepackage{amsmath}
\usepackage{bold-extra}

% time series countries colors
\definecolor{ts_belgium}{rgb}{0.11, 0.46, 0.70}
\definecolor{ts_armenia}{rgb}{0.99, 0.49, 0.5}
\definecolor{ts_austria}{rgb}{0.16, 0.62, 0.16}
\definecolor{ts_bulgaria}{rgb}{0.83, 0.14, 0.15}
\definecolor{ts_france}{rgb}{0.57, 0.40, 0.73}
\definecolor{ts_unitedstates}{rgb}{0.54, 0.33, 0.29}
\definecolor{ts_spain}{rgb}{0.88, 0.46, 0.75}
\definecolor{ts_germany}{rgb}{0.49, 0.49, 0.49}
\definecolor{ts_italy}{rgb}{0.73, 0.73, 0.12}
\definecolor{ts_brazil}{rgb}{0.0, 0.100, 0.84}

% bibliography references
\usepackage[backend=biber,style=numeric,sorting=none]{biblatex}
\addbibresource{main.bib}
\nocite{*}

\begin{document}

%-------------------------------------------------------------------------------
% Title
%-------------------------------------------------------------------------------
\begin{center}
	\huge{\bfseries{\scshape{COVID-19: Behind the Numbers}}}\\
	\vspace{1.0cm}
	\large{Data Mining and Machine Learning Project}\\
	\vspace{0.2cm}
	\large{Prof. Marcelloni Francesco}\\
	\vspace{0.2cm}
	\large{Prof. Ducange Pietro}\\
	\vspace{1.0cm}
	\large\textit{Rambod Rahmani}\\
	\vspace{0.2cm}
	\scriptsize{Master's Degree in Artificial Intelligence and
	Data Engineering}\\
	\vspace{1.0cm}
	\normalsize{\today}
\end{center}

%-------------------------------------------------------------------------------
% Table of contents
%-------------------------------------------------------------------------------
\tableofcontents
\newpage

%-------------------------------------------------------------------------------
% Section: Introduction
%-------------------------------------------------------------------------------
\section{Introduction}
\textbf{Decemebr 31, 2019}: \textit{Wuhan Municipal Health Commission, China,
reported a cluster of cases of pneumonia in Wuhan, Hubei Province. A novel
coronavirus was eventually identified.}\\
\textbf{January 1, 2020}: \textit{World Health Organization (WHO) had set up the
Incident Management Support Team across the three levels of the organization:
headquarters, regional headquarters and country level, putting the organization
on an emergency footing for dealing with the outbreak.}\\
\textbf{January 5, 2020}: \textit{WHO published the first Disease Outbreak News
on the new virus. This was a flagship technical publication to the scientific
and public health community as well as global media.}\\
\textbf{January 12, 2020}: \textit{China publicly shared the genetic sequence of
COVID-19.}\\
\\
At the beginning of 2020, a new virus started spreading around in the capital of
Central China's Hubei province: the city we all came to know as Wuhan. As it
turned out, this was the start of a world-changing event with overwhelming
extent: Coronavirus Disease 2019 (COVID-19). After the first wave of the virus
has passed over the entire world, the aim of this work is to address the
following questions:
\begin{itemize}
	\item Which countries have been affected the most by COVID-19?
	\item Which governments have taken the right actions to stop the
		spreading of the virus?
	\item Is it possible to build personalized predictive models for
		symptomatic COVID-19 patients based on basic health preconditions?
\end{itemize}
In order to fully answer these questions, first of all a reliable and big enough
dataset was needed. Second, Data Mining and Machine Learning techniques were
applied in order to obtain statistically significant results that could help
address the proposed questions. In the following pages the described work and
the resulting Python software is presented. The software architecture is
presented in the very last section in order to focus primarily on the dataset
retrieval and preprocessing, and on the analysis techniques and results.

%-------------------------------------------------------------------------------
% Section: Dataset
%-------------------------------------------------------------------------------
\section{Dataset}
Two different datasets were used:
\begin{itemize}
    \item The data on confirmed cases and confirmed deaths is updated daily and
    is published by Johns Hopkins University, the best available global dataset
    on the pandemic.
\end{itemize}
\subsection{Daily COVID-19 Data}
\subsubsection{Preprocessing}
\subsection{COVID-19 Medical Preconditions Data}
\subsubsection{Preprocessing}

%-------------------------------------------------------------------------------
% Section: Analysis
%-------------------------------------------------------------------------------
\section{Analysis}
As said in the introductory section, the analysis was carried out using data
mining and machine learning techniques in order to answer the proposed
questions. Each of the following subsections is focused on one of them.
\subsection{Which countries have been affected the most by COVID-19?}
To answer the very first question, we need to understand what is hidden behind
the official numbers and charts of confirmed COVID-19 active cases and deaths.
We are so used to watching them and sometimes we think we might even understand
how the COVID-19 pandemic is evolving as days goes by. For example, it is very
common to consider the following:
\begin{figure}[H]
    \begin{center}
        \hspace*{-1.8cm}
        \includegraphics[scale=0.44]{img/total-cases.pdf}
    \end{center}
    \vspace*{-0.4cm}
    \caption{Top 15 Countries Confirmed COVID-19 Cases}
\end{figure}
\noindent Is this really the right choice? Does this ranking tells us anything
meaningful about the current undergoing pandemic situation? From what we can
observe in Figure 1, clearly United States has higher confirmed COVID-19 cases
than countries such as Spain or Italy. Taking into account that the US is a much
bigger country, we can agree that this results do not imply that the US is more
affected than Spain, Italy or Germany. We can therefore think of a fairer
comparison independent of the country size: the number of infections needs to be
normalized to the population of each country:
\begin{figure}[H]
    \begin{center}
        \hspace*{-1.7cm}
        \includegraphics[scale=0.44]{img/total-cases-per-million.pdf}
    \end{center}
    \vspace*{-0.4cm}
    \caption{Top 15 Countries Confirmed COVID-19 Cases Per One Million Population}
\end{figure}
\noindent This provides a more coherent view of the countries most affected by
COVID-19, however it is not yet good enough: countries have different testing
policies and a more intensive COVID-19 testing rate gives more confirmed cases
while no testing at all would imply zero cases. We can all agree upon the fact
that zero cases with no testing at all does not really mean that a given country
is not affected by the pandemic. We therefore need a quantity unrelated to the
rate of testing.\\
This quantity is the number of deaths: this value is unbiased by the testing
rate. We will use the normalized number of daily deaths for comparing
countries.\\
\\
While the previous results were obtained using the original COVID-19 historical
dataset\cite{ourworldindata}, \textbf{this time some preprocessing and
resampling is needed}:
\begin{itemize}
    \item the daily values for some of the dates in the original dataset are missing:
    \begin{itemize}
        \item countries with too many missing values (more than $150$ days) were
        removed since no interpolation technique can really be effective in such
        cases;
        \item were few missing values were present, these were replaced using
        the mean;
    \end{itemize}
    \item plotting the data with daily granularity results in a time series plot
    which is not smooth and hard to read; the data was therefore resampled with
    weekly granularity;
\end{itemize}
To impute the missing values, i.e., to infer them from the known part of the
data, an univariate imputation algorithm was used: imputes values in the
i-th feature dimension using only non-missing values in that feature dimension.
To this end, the \texttt{SimpleImputer} class from the \texttt{scikit-learn}
Python package was used. Missing values were imputed with the mean of each
column.\\
\\
This is the resulting time series plot for some of the countries:
\begin{figure}[H]
    \begin{center}
        \hspace*{-1.7cm}
        \includegraphics[scale=0.46]{img/weekly-deaths.pdf}
    \end{center}
    \caption{COVID-19 Weekly Deaths Per One Million Population}
\end{figure}
\noindent Daily deaths count appears to be a fair measure to compare how hard
different countries have been hit by the virus. I addition, observing the plot
one can immediately point out that
\begin{itemize}
    \item some countries suffered more the first wave, some others the second
    and some both;
    \item countries such as {\color{ts_italy}Italy} and
    {\color{ts_belgium}Belgium} were devastated by the first wave and reacted
    by completely shutting down social life; yet they are not doing better with
    the second wave;
    \item countries such as {\color{ts_austria}Austria} and
    {\color{ts_germany}Germany} have not been hit hard by the first COVID-19
    wave; these states reacted fast and slowed down social life right at the
    beginning; as a result, the number of deaths went back to almost zero; yet
    they are suffering from the second wave;
    \item countries such as {\color{ts_unitedstates}United States} and
    {\color{cyan}Brazil} have a daily death toll with and almost constant
    trend.
\end{itemize}
If we extend the same reasoning to all the countries we have data for:
\begin{figure}[H]
    \begin{center}
        \hspace*{-1.7cm}
        \includegraphics[scale=0.46]{img/weekly-deaths-worldwide.pdf}
    \end{center}
    \caption{COVID-19 Weekly Deaths Per One Million Population Worldwide}
\end{figure}
\noindent Figure 4 shows the weekly deaths per one million population for 190
countries, from Afghanistan to Zimbabwe. It is humanly impossible to extract
any useful information from such a plot. In this mess, how many different
characteristic curves can we find? To clean the mess, find patterns and extract
the required knowledge, a clustering algorithm was used. This unsupervised
learning technique groups similar data curves together.


To summarize, by means of a machine learning algorithm applied to the COVID-19
data we organized countries into groups with similar epidemiological behavior.
Surprisingly, those groups form local clusters on the world map as well. This
unexpected insight helps us to answer the question proposed in this section.

\subsection{Which governments have taken the right actions?}
\subsection{Personalized predictive models for symptomatic COVID-19 patients}
%-------------------------------------------------------------------------------
% Section: Conclusion
%-------------------------------------------------------------------------------
\section{Conclusion}

%-------------------------------------------------------------------------------
% Section: Software Architecture
%-------------------------------------------------------------------------------
\section{Software Architecture}

\printbibliography

\end{document}
